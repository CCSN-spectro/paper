% !TEX root = ccsn.tex
\section{Conclusions}
\label{sec:conclusion}

The detection of GW from CCSN may help improve our current understanding of the explosion mechanism of massive stars.  
In this paper we have proposed an explorary method to infer PNS properties using an approach based on GW associated with convective oscillations of PNS. As shown by~\cite{Torres:2019b} buoyancy-driven g-modes are excited in numerical simulations of CCSN and their time-frequency evolution is linked to the physical properties of the compact remnant through universal relations. Such modes are responsible for a significant fraction of the highly stochastic GW emitted after core bounce. The findings reported in this paper suggest that PNS asteroseismology might be within reach of current and third-generation GW detectors.

In our study we have used a set of 1D CCSN simulations to build a model that relates the  evolution of PNS properties with the frequency of the dominant g-mode, namely the $\mbox{}^2g_2$ mode. This relationship is extracted from the GW data using an algorithm developed for this investigation. This algorithm is  a first attempt to infer the time evolution of a particular combination of the PNS mass and radius based on the universal relations found in~\cite{Torres:2019b}. More precisely, we have considered the ratio $r=M_{\rm PNS}/R_{\rm PNS}^2$ (the PNS surface gravity) derived from the observation of the $\mbox{}^2g_2$ oscillation mode in the numerically generated GW data. The  performance of our method has been estimated employing  simulations of 2D CCSN waveforms covering a progenitor mass range between 11 and 40 solar masses and different equations of state. 

We have investigated the performance of the algorithm in the case of an optimally oriented source detected by a singe GW detector. Our numerical signals have been injected into 100 Gaussian noise realisations whose PSD follow the spectra of the different GW detectors analyzed. We have found that for Advanced LIGO and Advanced Virgo, the ratio $r$ can be reconstructed with a good accuracy in the case of a galactic CCSN (i.e.~for distances of ${\cal O}(10\, {\rm kpc})$). This holds for a wide range of progenitor masses, the quality of the inference mainly depending on the signal-to-noise ratio of the event. For third-generation GW detectors such as the Einstein Telescope and the Cosmic Explorer, however, we obtain an order of magnitude improvement, as the $\mbox{}^2g_2$ ratio can be reconstructed for sources at distances of ${\cal O}(100\, {\rm kpc})$. In particular, Cosmic Explorer in its stage 2 configuration yields the best performance for all waveforms we have considered thanks to its excellent sensitivity in the \unit[100-1000]{Hz} range. Among the three configurations of the Einstein Telescope, ET-D provides the best performance, especially for our set waveforms with the highest progenitor masses (25 $\Msol$ and 40 $\Msol$). Comparing the estimated distances for ET-B and the other third-generations detectors, having a good sensitivity below \unit[200]{Hz} seems the most important factor to detect high mass progenitor signals.

In the present study we have assumed that the sources are optimally oriented. The reported distances at which
we can infer the time evolution of $M_{\rm PNS}/R_{\rm PNS}^2$ must thus be regarded as upper limits. Those figures may decrease by a factor 2--3 on average for a source located anywhere in the sky. Furthermore, we have not considered the detectability prospects of  CCSN waveforms in the realistic case in which the  interferometers operated as a detector network. 
We defer an improved implementation of our approach for a forthcoming publication. Finally, we note that the method discussed in this work can be adapted to other PNS oscillation modes, by simply changing a few parameters such as the initial frequency range of the mode and its monotonic raise or descent. Being able to reconstruct several modes in the same GW signal would potentially allow to individually infer the mass and the radius of the PNS in core-collapse supernova explosions.


%Furthermore, the results presented here are based on a model of the frequency evolution of the
%ratio that is making use of AENUS-ALCAR code simulations. Using simulations produced
%with the CoCoNut code, the model fit is a bit different (see Figure \ref{fig:LMVAR}). This systematic
%difference of the two simulation codes is discussed in \cite{Torres:2019b}. If we consider a model based
%on CoCoNut simulations and test the method on test simulations obtained with the AENUS-ALACR code,
%we observe systematic effects in the ratio reconstruction leading to systematically worsen performance.
%As the goal of this paper is to characterize the methods and because both simulation codes have pros
%and cons we have reported results obtained with a model set and test set waveforms obtained with the same
%code.


