\section{Conclusion}
The algorithm presented in this paper is a first attempt to infer the time evolution of a
combinaison of the mass of the PNS and its radius based on the universal relations found
in PNS asteroseismology. More precisely, we have considered in this paper the ratio
$r=M_{\rm PNS}/R_{\rm PNS}^2$ derived from the observation of the $\mbox{}^2g_2$
oscillation mode in the GW data. We have especially investigated the performance of the algorithm
in the case of an optimally oriented source detected in a singe GW detector. For Advanced LIGO
or Advanced Virgo, the ratio can be reconstructed for a source in the Galaxy. We have shown
that this is true for a wide range of progenitor masses and that the quality of the inference
mainly depends on the signal-to-noise ratio of the signal. For third generation of GW detectors such
as Einstein Telescope and Cosmic Explorer, the $\mbox{}^2g_2$ will be reconstructible for sources
at distances of several kpc. Cosmic Explorer in its stage 2 configuration is obtaining the best performance
for all waveforms considered here thanks to its excellent sensitivity in the \unit[100-1000]{Hz} range.
Among the three configuration of Einstein Telescope, ET-D is providing the best performance,
especially for the waveforms with the highest progenitor mass (25 $\Msol$ and 40 $\Msol$). Comparing
$d_r$ for ET-B and the other third generations projects, it seems that having a good sensitivity
below \unit{200}[Hz] is important for massive mass progenitor signals.

%The model to infer the ratio as function of the frequency is based on 1D {\sc AENUS-ALCAR} simulations.
%We have observed that using the other numerical code {\sc CoCoNut} inputs lead to systematic bias of the
%ratio reconstruction. 

This study does not include the realistic case of operating within a network of detectors. We defer this
for a forthcoming publication. The sources of GWs we have considered here are optimally oriented. The
reported distance at which we can infer the time evolution of $r=M_{\rm PNS}/R_{\rm PNS}^2$ are thus an
upper limit that may be lower by a factor 2--3 on average for a source located anywhere on the sky.

Finally, this method can be adapted to other PNS oscillation modes, changing few parameters such as
the frequency range of the beginning of the mode and its monotonic raise or descent. Being able to
reconstruct several modes in the same GW signal would allow to infer individually each of the PNS property.

