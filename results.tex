\section{Detection sensitivity with Advanced gravitational wave detectors}
\label{sec:results}

To estimate how accurately we can infer the time evolution of $r=M_{\rm PNS}/R_{\rm PNS}^2$ in the
gravitational wave detector data, we have added the {\tt s20-gw-10kpc} GW signal to hundred
Gaussian noise realisations whose power spectral density follows advanced LIGO
spectrum~\cite{aLIGOspectrum}. We have varied the distances to the source, covering a large
range of distances for which a detection is feasible. The source is optimally oriented with
respect to the gravitational wave detector. We are assuming a GW signal from a core collapse
phenomena has been identified in the data and that the beginning of the GW signal is known.
For each of the simulations, we reconstruct the ratio and compute few quantities that measure
the reconstruction accuracy. The first quantity is the coverage probability which is the
fraction of the ratio that falls within the 95\% confiddence interval.  


For that purpose, we inject the gravitational wave signal into  simulated Advanced LIGO noise
using the noise power spectral density \textcolor{red}{insert formula} for varying
SNRs, respectively distances to the source. We estimate the coverage probability of the 95\%
confidence band by calculating the proportion of times that the true ratio lies outside one
of the pointwise 95\% confidence intervals.
These coverage probabilities together for varying SNRs are given in Table
\textcolor{red}{insert Table} and displayed in the form of boxplots in Figure
\textcolor{red}{insert Figure}
