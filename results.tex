\section{Detection sensitivity with Advanced gravitational wave detectors}
\label{sec:results}

To estimate how accurately we can infer the time evolution of $r=M_{\rm PNS}/R_{\rm PNS}^2$ in the
gravitational wave detector data, we have added {\texttt s20S} GW signal to 
Gaussian noise realisations whose power spectral density follows advanced LIGO
spectrum~\cite{aLIGOsens:2018} shown on Figure~\ref{fig:spectrum}. 
We have varied the distance to the source, covering a large
range of distances for which a detection in second generation of gravitational wave detectors
is feasible. The source is optimally oriented with
respect to the gravitational wave detector. We are assuming a GW signal from a core collapse
phenomena has been identified in the data and that the beginning of the GW signal is known within $O(10~ms)$.
The data (signal embedded in noise) are whitened using the function {\it prewhiten} of the R-package TSA.
An auto-regressive model with maximal \textcolor{red}{100} coefficients has been used.    

For each of the noise realisations, we reconstruct the ratio time series {$r_i$}
of length $N$ and compute two quantities that compare {$r_i$} to ratio {$r_i^0$} derived from
the PNS mass and radius generated by the simulation code that produces {\texttt s20S}.

The first quantity, $coverage$, is the fraction of the
ratio {$r_i^0$} values that fall within the 95\% confidence interval of {$r_i$}.
We also compute the $precision$ value given by
\begin{equation}
precision=\sum_1^N\frac{|r_i-r_i^0|}{r_i^0}
\end{equation}

Figure \ref{fig:s20results} is showing the median of $coverage$ and $precision$ as
function of the distance of the source as well as the confidence belt corresponding
to the median absolute deviation. As expected, the estimation of $r$ is maximal when
the source is nearby and decreases with the distance. At small distance the $precision$
value is small but not null. This reflects the approximtaion of the model used for $r$.
It is nevertheless remarkable that one can reconstruct the ratio time series with a good
precision at distance up to $\sim$ 10 kpc for this particular waveform. We have tested
that the method does not depend on features of {\texttt s20S} using 7 other waveforms
described in section \ref{sec:simulation} covering a large range of progenitor masses.
Figure \ref{fig:aLIGOall} shows that apart {\tt s11.2--LS220}, the ratio is well
reconstructed for all waveforms up to $\sim$ 10kpc. 

\begin{figure}
  \centering
  \includegraphics[width=0.5\textwidth]{plots/s20-gw_covpbb_prec_aLIGO}
 \caption{$coverage$ and $precision$ for {\texttt s20S} signal embedded in aLIGO noise at different distance from the Earth. The shaded regions are given by the median absolute deviation.} \label{fig:s20results}
\end{figure}


\begin{figure}
 \centering
 \includegraphics[width=0.5\textwidth]{plots/spectrum}
 \caption{} \label{fig:spectrum}
\end{figure}

\begin{figure}
  \centering
  \includegraphics[width=0.5\textwidth]{plots/covppb_all_aLIGO}
 \caption{$coverage$ for 7 waveforms embedded in aLIGO noise at different distance from the Earth. } \label{fig:aLIGOall}
\end{figure}


\begin{table}
  \centering
  \begin{tabular}{c|cccccccc}
    \hline
    
    Simulation & \texttt{s11} & \texttt{s15} & \texttt{s15S} & \texttt{s15G} & \texttt{s20} & \texttt{s20S} & \texttt{s25}  & \texttt{s40}
    \\   
    \hline
    aLIGO      & 19.5         & 55.4         & 59.0          &   60.0        & 34.3         &   35.8          & 116.5         & 98.5
    \\ 
    \hline
    CE2
    \\
    \hline
    ET-B
    \\ 
  \end{tabular}
  \caption{%%
    Matched filter signal-to-noise ratio (SNR) of the simulated waveforms for
    the different GW detectors considered in this study. The source is located
    at 10 kpc and is optimally oriented with respect to the detector.
    %%
  }
  \label{Tab:SNR}
\end{table}
