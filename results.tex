\section{Detection sensitivity with Advanced gravitational wave detectors}
\label{sec:results}

To estimate how accurately we can infer the time evolution of $r=M_{\rm PNS}/R_{\rm PNS}^2$ in the
gravitational wave detector data, we have added {\tt s20-gw-10kpc} GW signal to 
Gaussian noise realisations whose power spectral density follows advanced LIGO
spectrum~\cite{aLIGOsens:2018}. 
We have varied the distances to the source, covering a large
range of distances for which a detection in second generation of gravitational wave detectors
is feasible. The source is optimally oriented with
respect to the gravitational wave detector. We are assuming a GW signal from a core collapse
phenomena has been identified in the data and that the beginning of the GW signal is known.
The data (signal embedded in noise) are whitened using the function {\it prewhiten} of the R-package TSA.
An auto-regressive model with maximal \textcolor{red}{100} coefficients has been used.    

For each of the noise realisations, we reconstruct the ratio time series {$r_i$}
of length $N$ and compute two quantities that compare {$r_i$} to ratio {$r_i^0$} derived from
the PNS mass and radius generated by the simulation code that produces {\tt s20-gw-10kpc}.
measure the estimation accuracy.
The first quantity is the coverage probability ($covpbb$) which is the fraction of the
ratio {$r_i^0$} values that fall within the 95\% confidence interval of {$r_i$}.
We also compute the $precision$ value given by
\begin{equation}
precision=\sum_1^N\frac{|r_i-r_i^0|}{r_i^0}
\end{equation}

Figure \ref{fig:s20results} is showing the median of $covpbb$ and $precision$ as
function of the distance of the source as well as the confidence belt corresponding
to the median absolute deviation.

\begin{figure}
  \centering
  \includegraphics[width=0.5\textwidth]{plots/s20-gw_10kpc16384_covpbb_prec}
 \caption{$covpbb$ and $precision$ for {\tt s20-gw-10kpc} signal embedded in aLIGO noise at different distance from the Earth. The shaded regions are given by the median absolute deviation.} \label{fig:s20results}
\end{figure}



