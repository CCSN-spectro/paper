% !TEX root = ccsn.tex
\section{Core collapse supernova simulations}
\label{sec:simulations}

Unlike other methods used in GW astronomy the algorithm proposed in {this work} does not require accurate
waveforms {in order to infer the properties of the PNS.} Instead, it relies on the evolution of the oscillation
frequency of some particular modes visible in the GW spectrum.
The frequency of these modes depends in a universal way on the surface gravity of the PNS, $r\equiv M_{\rm PNS}/R_{\rm PNS}^2$ \citep{Torres:2019b}. Therefore, if at a given time GW emission is observed at a certain
frequency $f$ then the value of the surface gravity can be determined unequivocally, within a certain error,
regardless of the details of the numerical simulation. 

In this work we use two sets of simulations: i) the {\it model set}, composed by 1D simulations, which is used to build the 
universal relation (model), $r(f)$, linking the ratio $r$ with the observed frequency $f$, and ii) the {\it test set}, composed by
2D simulations, 
for which we know both the GW signal and the evolution of the ratio, $r (t)$, and that is used to test
performance of the algorithm.

%{We have used two different numerical codes in our numerical simulations.} 
%CoCoNuT
%(one-dimensional models) and AENUS-ALCAR
%\citep{Just_et_al__2015__mnras__Anewmultidimensionalenergy-dependenttwo-momenttransportcodeforneutrino-hydrodynamics}
%(one- and two-dimensional models). 
%CoCoNuT
%\citep{Dimmelmeier:2002,Dimmelmeier:2005} is a code for general
%relativistic hydrodynamics coupled to the Fast Multigroup Transport
%scheme \citep{Mueller_Janka_2015_FMT} providing an approximate
%description of the emission and transport of neutrinos.
Both the {\it model set} and {\it test set} simulations have been generated using the AENUS-ALCAR code
\citep{Just_et_al__2015__mnras__Anewmultidimensionalenergy-dependenttwo-momenttransportcodeforneutrino-hydrodynamics}
which combines special relativistic (magneto-)hydrodynamics, a modified
Newtonian gravitational potential approximating the effects of general
relativity \citep{Marek_etal__2006__AA__TOV-potential}, and a spectral
two-moment neutrino transport solver
\citep{Just_et_al__2015__mnras__Anewmultidimensionalenergy-dependenttwo-momenttransportcodeforneutrino-hydrodynamics}.
All siimulations include the relevant reactions between matter and neutrinos of all
flavours, i.e., emission and absorption by nucleons and nuclei,
electron-positron pair annihilation, nucleonic bremsstrahlung, and
scattering off nucleons, nuclei, and electrons.

For the {\it model set}, we use the $25$ spherically symmetric (1D) simulations of \citep{Torres:2019a}
including progenitors with zero-age main sequence (ZAMS) masses in the range 
$M_{\mathrm{ZAMS}} = 11.2 - 75 \, \Msol$. The set contains simulations using  
six different EOS. Details can be found in
 \citep{Torres:2019a}. The reason to use 1D simulations for the model set
 is that their computational cost is significantly smaller than that of multidimensional
 simulations which allows us to accumulate the statistics necessary to build a good model for $r(f)$.
 {For each time of each simulation we compute the ratio $r$ and the frequency of the $^2g_2$ mode by means of the linear analysis
 described in \cite{Torres:2018,Torres:2019a,Torres:2019b}. } 
 
 \begin{table}
 \centering
 \begin{tabular}{c|ccc|ccc}
  \hline
  Model & $M_\mathrm{ZAMS} $ & progenitor& EOS & $t_{\mathrm{f}}$& $t_{\rm explosion}$ & $M_{\mathrm{PNS, f}}$\\
  name& $[\Msol]$ & model & & $[\mathrm{s}]$& & $[\Msol]$ 
  \\ 
  \hline
  \texttt{s11} & 11.2 & \cite{Woosley_Heger_Weaver__2002__ReviewsofModernPhysics__The_evolution_and_explosion_of_massive_stars}& LS220 & 1.86 & $\times$ & 1.47 
  \\ 
  \texttt{s15} & 15.0 & \cite{Woosley_Heger_Weaver__2002__ReviewsofModernPhysics__The_evolution_and_explosion_of_massive_stars}& LS220 & 1.66 & $\times$ & 2.00 
    \\ 
  \texttt{s15S} & 15.0 & \cite{Woosley_Heger_Weaver__2002__ReviewsofModernPhysics__The_evolution_and_explosion_of_massive_stars}& SFHo & 1.75 & $\times$ & 2.02 
    \\ 
  \texttt{s15G} & 15.0 & \cite{Woosley_Heger_Weaver__2002__ReviewsofModernPhysics__The_evolution_and_explosion_of_massive_stars}& GShen & 0.97 & $\times$ & 1.86
     \\ 
  \texttt{s20} & 20.0 & \cite{Woosley_Heger_Weaver__2002__ReviewsofModernPhysics__The_evolution_and_explosion_of_massive_stars}& LS220 & 1.53 & $\times$ & 1.75 
    \\ 
  \texttt{s20S} & 20.0 & \cite{Woosley_Heger__2007__physrep__Nucleosynthesisandremnantsinmassivestarsofsolarmetallicity} & SFHo & 0.87 & $\times$ & 2.05 
  \\ 
  \texttt{s25} & 25.0 & \cite{Woosley_Heger_Weaver__2002__ReviewsofModernPhysics__The_evolution_and_explosion_of_massive_stars}& LS220 & 1.60 & $0.91$ & 2.33 
    \\ 
  \texttt{s40} & 40.0 & \cite{Woosley_Heger_Weaver__2002__ReviewsofModernPhysics__The_evolution_and_explosion_of_massive_stars}& LS220 & 1.70 & $1.52$ & 2.23 
    \\ \hline
 \end{tabular}
 \caption{%%
  List of axisymmetric simulations {used for the {\it test set}}. 
  {The last three columns show, the post-bounce time at the end of the
  simulation, the one at the onset of the explosion (non exploding models marked
  with $\times$), and the PNS mass at the end of the simulation.}
  %%
 }
 \label{Tab:2dSimList}
\end{table}

{For the {\it test set}, we use $8$ axisymmetric (2D) simulations performed with the {\sc AENUS-ALCAR} code
(see Table~\ref{Tab:2dSimList} for a list of models).
All of these simulations but model \texttt{s20S} use a selection of progenitors with masses in the range} $M_{\mathrm{ZAMS}} = 11.2 - 40 \, \Msol$
 evolved through the hydrostatic phases by
\cite{Woosley_Heger_Weaver__2002__ReviewsofModernPhysics__The_evolution_and_explosion_of_massive_stars}.
We performed one simulation of each stellar model using the EOS of \cite{Lattimer_Swesty__1991__NuclearPhysicsA__LS-EOS} with
an incompressibility of $K = 220 \, \mathrm{MeV}$ (LS220) and added
comparison simulations with the SFHo EOS
\cite{Steiner_et_al__2013__apj__Core-collapseSupernovaEquationsofStateBasedonNeutronStarObservations}
and the GShen EOS 
\cite{Shen_et_al__2011__prc__Newequationofstateforastrophysicalsimulations}
for the {progenitor} with $M_{\mathrm{ZAMS}} = 15 \, \Msol$. To this set of
simulations we add the waveform of a 2D model used in
\cite{Torres:2019a}, denoted \texttt{s20S}. It corresponds to a star
with the same initial mass, $M_{\mathrm{ZAMS}} = 20 \, \Msol$, as for
one of the other seven axisymmetric simulations, but was taken from a
newer set of stellar-evolution models
\cite{Woosley_Heger__2007__physrep__Nucleosynthesisandremnantsinmassivestarsofsolarmetallicity}.
It was evolved with the SFHo EOS.


{For all the simulations,} we mapped the pre-collapse state of the stars to a spherical
coordinate system with $n_r = 400$ zones in the radial direction
distributed logarithmically with a minimum grid width of
$(\Delta r)_{\mathrm{min}} = 400 \, \mathrm{m}$ and an outer radius of
$r_{\mathrm{max}} = 8.3 \times 10^{9} \, \mathrm{cm}$ and
$n_{\theta} = 128$ equidistant cells in the angular (polar) direction. For the
neutrino energies, we used a logarithmic grid with $n_e = 10$ bins up
to $240 \, \mathrm{MeV}$.
{Unlike the model set, the simulations in the test set are not 1D because we need to 
extract the GW signal, which is a multi-dimensional effect. For each simulation
the GW signal, $h_+(t)$, is extracted by means of the quadrupole formula and we compute the 
time evolution of the surface gravity, $r(t)$.}

All spherical and most axisymmetric models we evolved fail to achieve shock
revival during the time of our simulations. Only the two stars with
the highest masses, \texttt{s25} and \texttt{s40}, develop relatively
late explosions in axisymmetry. Consequently, mass accretion onto the
PNSs proceeds at high rates for a long time in all cases and causes
them to oscillate with their characteristic frequencies. The final
masses of the PNSs are in the range of
$M_{\mathrm{PNS}} = 1.47 - 2.33 \, \Msol$, i.e., likely insufficient for
producing a black hole.


 	

%%% Local Variables:
%%% TeX-master: "ccsn"
%%% End:
