\section{Core collapse supernova simulations}
\label{sec:simulations}

% \textbf{Martin's simulations and code description}\\
% \textbf{1D simlation data to fit the ratio vs frequency model - AA and CoConut outputs}


The algorithm proposed in the article does not require accurate
waveforms but relies on the evolution of the frequencies of
oscillations depending on the PNS mass and radius.  To parametrize
this dependence, we have considered spherically symmetric
\citep{Torres:2019a} and two-dimensional axisymmetric models of
stellar core collapse simulated with two codes, CoCoNuT
(one-dimensional models) and AENUS-ALCAR
\citep{Just_et_al__2015__mnras__Anewmultidimensionalenergy-dependenttwo-momenttransportcodeforneutrino-hydrodynamics}
(one- and two-dimensional models).  CoCoNuT
\citep{Cerda-Duran__2008__AA__GRMHD-code} is a code for general
relativistic hydrodynamics coupled to the Fast Multigroup Transport
scheme \citep{Mueller_Janka_2015_FMT} providing an approximate
description of the emission and transport of neutrinos.  AENUS-ALCAR
\citep{Just_et_al__2015__mnras__Anewmultidimensionalenergy-dependenttwo-momenttransportcodeforneutrino-hydrodynamics}
combines special relativistic (magneto-)hydrodynamics, a modified
Newtonian gravitational potential approximating the effects of general
relativity \citep{Marek_etal__2006__AA__TOV-potential}, and a spectral
two-moment neutrino transport solver
\citep{Just_et_al__2015__mnras__Anewmultidimensionalenergy-dependenttwo-momenttransportcodeforneutrino-hydrodynamics}.
We included the relevant reactions between matter and neutrinos of all
flavours, i.e., emission and absorption by nucleons and nuclei,
electron-positron pair annihilation, nucleonic bremsstrahlung, and
scattering off nucleons, nuclei, and electrons.

We use two sets of 25 models in the range of initial stellar masses
$M_{\mathrm{ZAMS}} = 11.2 - 75 \, \Msol$ simulated with the two codes.
They were carried out using six equations of state (EOSs).
In addition to these simulations data we have considered 8 waveforms.
7 of them are from two-dimensional axisymmetric models
consisting of stellar core collapse of five stars with zero-age
main-sequence masses of $M_{\mathrm{ZAMS}} = 11.2 - 40 \, \Msol$
evolved through the hydrostatic phases by
\cite{Woosley_Heger_Weaver__2002__ReviewsofModernPhysics__The_evolution_and_explosion_of_massive_stars}.
We performed one simulation of each stellar model using the equation
of state of \cite{Lattimer_Swesty__1991__NuclearPhysicsA__LS-EOS} with
an incompressibility of $K = 220 \, \mathrm{MeV}$ (LS220) and added
comparison simulations with the SFHo EOS
\cite{Steiner_et_al__2013__apj__Core-collapseSupernovaEquationsofStateBasedonNeutronStarObservations}
and the EOS of
\cite{Shen_et_al__2011__prc__Newequationofstateforastrophysicalsimulations}
(GShen) for the one with $M_{\mathrm{ZAMS}} = 15 \, \Msol$ (see Table
\ref{Tab:2dSimList} for a list of models).  To this set of
simulations, we add the waveform of a two-dimensional model used in
\cite{Torres:2019a}, denoted \texttt{s20S}.  It corresponds to a star
with the same initial mass, $M_{\mathrm{ZAMS}} = 20 \, \Msol$, as for
one of the other 7 axisymmetric simulations, but was taken from a
newer set of stellar-evolution models
\cite{Woosley_Heger__2007__physrep__Nucleosynthesisandremnantsinmassivestarsofsolarmetallicity}.
It was evolved with the SFHo EOS.

We mapped the pre-collapse state of the stars to a spherical
coordinate system with $n_r = 400$ zones in radial direction
distributed logarithmically with a minimum grid width of
$(\Delta r)_{\mathrm{min}} = 400 \, \mathrm{m}$ and an outer radius of
$r_{\mathrm{max}} = 8.3 \times 10^{9} \, \mathrm{cm}$ and
$n_{\theta} = 128$ equidistant cells in angular direction.  For the
neutrino energies, we used a logarithmic grid with $n_e = 10$ bins up
to $240 \, \mathrm{MeV}$.

All spherical and most axisymmetric models fail to achieve shock
revival during the time of our simulations.  Only the two stars with
the highest masses, \texttt{s25} and \texttt{s40}, develop relatively
late explosions in axisymmetry.  Consequently, mass accretion onto the
PNSs proceeds at high rates for a long time in all cases and causes
them to oscillate with their characteristic frequencies.  The final
masses of the PNSs are in the range of
$M_{\mathrm{PNS}} = 1.47 - 2.33 \, \Msol$, i.e., insufficient for
producing a black hole.

\begin{table}
  \centering
  \begin{tabular}{c|cc|ccc}
    \hline
    Simulation & $M_\mathrm{ZAMS} [\Msol]$ & EOS & $t_{\mathrm{f}} [\mathrm{s}]$&
    explosion & $M_{\mathrm{PNS}} [\Msol]$
    \\ 
    \hline
    \texttt{s11} & 11.2 & LS220 & 1.86 & $\times$ &  1.47
    \\ 
    \texttt{s15} & 15.0 & LS220 & 1.66 & $\times$ &  2.00
    \\ 
    \texttt{s15S} & 15.0 & SFHo & 1.75 & $\times$ &  2.02
    \\ 
    \texttt{s15G} & 15.0 & GShen & 0.97 & $\times$ &  1.86
    \\ 
    \texttt{s20} & 20.0 & LS220 & 1.53 & $\times$ & 1.75
    \\ 
    \texttt{s20S} & 20.0 & SFHo & 0.87 & $\times$ & 2.05
    \\ 
    \texttt{s25} & 25.0 & LS220 & 1.60 & $0.91$ &  2.33
    \\ 
    \texttt{s40} & 40.0 & LS220 & 1.70 & $1.52$ &  2.23
    \\ \hline
  \end{tabular}
  \caption{%%
    List of axisymmetric simulations.  We present the name of the models, the initial
    mass of the progenitors, and the EOS used, and the  final
    post-bounce time of the simulations.  For models which explode,
    we list the time at which the shock starts to expand in column
    ``explosion''; otherwise, a $\times$ sign is displayed.  The final
    column indicates the mass of the PNS at the end of the
    simulation.  
    %%
  }
  \label{Tab:2dSimList}
\end{table}

 	

%%% Local Variables:
%%% TeX-master: "ccsn"
%%% End:
