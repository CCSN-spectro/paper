% !TEX root = ccsn.tex

\section{Introduction}


The life of sufficiently massive stars, i.e.~those born with masses between $\sim 8$~M$_\odot$ and $\sim 120$~M$_\odot$, ends with the collapse of {the} iron core under {its} own gravity, leading {to} the formation of a neutron star {(NS)} or a black hole (BH), {and} followed (typically but not necessarily in the BH case) by {a supernova} explosion. Nearby core-collapse supernova (CCSN) explosions are expected to be sources of gravitational waves (GWs) and they could be 
%are one of the main candidates for 
the next great discovery of current ground-based observatories. However, these are relative rare events. A neutrino-driven explosion \citep{Bethe:1990} is the most likely outcome in the case of slowly rotating cores, which are present in the bulk of CCSN progenitors. The emitted GWs could be detected with the advanced ground-based GW detector network (Advanced LIGO~\citep{TheLIGOScientific:2014jea}, Advanced Virgo~\citep{TheVirgo:2014hva} and
KAGRA~\citep{Aso:2013eba}) within $5$~kpc \citep{Gossan:2016,TargetedSNSearchO12}. Such a galactic event has a rate of about $2-3$ per century \citep{Adams:2013,Rozwadowska:2021}.
For the case of rapidly rotating progenitor cores the result is likely a magneto-rotational explosion, yielding  a more powerful GW signal that could be detected within $50$~kpc and, for some extreme models, up to $5-30$ Mpc \citep{Gossan:2016,TargetedSNSearchO12}. However, only about $1\%$ of the electromagnetically observed events show signatures of fast rotation (broad-lined type Ic SNe \citep{Li:2011b} or events associated with long GRBs 
\citep{Chapman:2007}) making this possibility a subdominant channel of detection with an event rate of $\sim 10^{-4} \rm{yr}^{-1}$\mab{[add ref?]}. For the results discussed in this work we only consider neutrino-driven CCSN. \tf{This last sentence must be mentioned somewhere (or maybe not) but perhaps here is not the best place.} Despite the low rates, CCSN are of great scientific interest because they produce complex GW signals which could provide significant clues about the physical processes at work after the gravitational collapse of stellar cores. 

In the last decade significant progress has been made in the development of numerical codes, {in particular in the treatment of multidimensioal effects \citep{BMueller:2020}.} In the case of  neutrino-driven explosions, the GW emission is {primarly induced by instabilities developed at the newly formed proto-neutron star (PNS) and by the non-spherical accreting flow of hot matter over its surface \citep{Kotake:2017}.  This} dynamics excite the different modes of oscillation of the PNS which ultimately leads to the emission of GWs. The frequency and time evolution of these modes carry information about the properties of the GW emitter and could allow to perform PNS asteroseismology. 


%{[PCD: I would remove the discussion on the inference of the progenitor properties (commented now). ]}
%Unluckily, in this case it is not posible to relate the GW emission with the properties (mass, rotation rate, metallicity or magnetic fields) of the progenitor stars.  A large number of physical processes are involved and their role is not completely understood. For instance,  uncertainties in the stellar evolution models of massive stars or in the nuclear and weak interactions necessary for the equation of state (EoS) of nuclear matter or the neutrino interactions. Furthermore, the stochastic and chaotic nature of the instabilities is transferred to the GW emission, resulting in the same progenitor leading significantly different waveforms.
%The large number of physical ingredients in addition to the necessary accuracy of the modelling of complex multidimensional interactions requires large computational resources. One simulation of a single progenitor explosion  in 3D with accurate neutrino transport and realistic EoS can take several months of intense calculations on a scientific supercomputer facility. This complicates the systematic exploration of the progenitor parameters.

All multidimensional numerical simulations show the systematic appearance in time-frequency diagrams (or spectrograms) of a distinct and relatively narrow feature 
%
%{The main feature appearing systematically in the GW spectrum of multidimensional numerical simulations is a strong and relatively narrow \mab{oscillation} 
%
during the post-bounce evolution of the system, with raising frequency
from about $100$~Hz up to a few kHz (at most) and a typical duration of $0.5-1$~s. This feature has been interpreted as a continuously excited gravity mode (g-mode, see \citep{kokkotas,Friedman:2013} for a definition in this context) of the PNS \citep{murphy:09, mueller:13gw, Cerda:2013, Yakunin:2015, Kuroda:2016, Andresen:2017}. 
In these models the monotonic raise of the frequency of the mode is related to the contraction of the PNS. The {typical} frequencies of {these} modes make them interesting targets for
%promising {source} for 
ground-based GW interferometers. 
 
 The {properties of} g-modes in hot {PNSs} have been studied since the 1990s
 %end of last century 
 {by means of linear perturbation analysis of background PNS models}. The oscillation modes connected with the surface of hot PNSs were first considered by McDermott {\it et al.} \citep{McDermott:1983}. Additionally, the stratified structure of the PNS allows for the presence of different types of g-modes related to the fluid core \citep{Reisenegger:1992}. Many subsequent works used simplified neutron star models assuming an equilibrium configuration {as a background}, to study the effect of rotation \citep{Ferrari:2004}, general relativity \citep{Passamonti:2005}, non-linearities \citep{Dimmelmeier:2006}, phase transitions \citep{Kruger:2015} and realistic equation of state \citep{Camelio:2017}. {Only recently, there have been efforts to incorporate more suitable backgrounds based on numerical simulations in the computation of the mode structure and evolution \citep{Sotani:2016,Torres:2018, Morozova:2018, Torres:2019a,Torres:2019b,Sotani:2019,WS:2019,Sotani:2020a, Sotani:2020b}}.
 
Using results from 2D CCSN numerical simulations as a background with respect to which to solve an eigenvalue problem~\citep{Torres:2018, Torres:2019a} found that the eigenmode spectrum of the region within the shock (including the PNS and the post-shock region) 
shows a good match to the mode frequencies and to the features observed in the GW spectrum of the same simulations (specially when space-time perturbations are included \citep{Torres:2019a}). \tf{This last sentence might be improved.}
This reveals that it is posible to perform CCSN asteroseismology {under realistic conditions} and serves as a starting point to carry out inference of astrophysical parameters of PNSs. 
Further work was presented in {\citep{Torres:2019b} who found that it is possible to derive simple relations} between the instantaneous frequency of the g-mode and the mass and radius of the PNS {at each time of the numerical evolutions}. These relations are universal as they do not depend {on the equation of state (EOS) or on the mass of the progenitor {and they only depend weakly on} the numerical code used {(see discussion in Section~\ref{sec:simulations})}. {Similar universal relations have been discussed by \citep{Sotani:2020a,Sotani:2020b} who also found that they do not depend on the dimensionality (1D, 2D or 3D) of the numerical simulation used as a background.
 
{In this work we introduce a method to infer PNS properties, namely a combination of the mass and radius, using GW information. For this purpose we have developed an algorithm to  
extract the time-frequency evolution of the main feature in the spectrograms of the GW emission of 2D simulations of CCSN. This feature corresponds to the $^2\rm{g}_2$ mode, according to the nomenclature used in \citep{Torres:2019b} (different authors may have slightly different naming convention). Next, we use the universal relations obtained by \citep{Torres:2019b}{, based on a set of 1D simulations,} to infer the time evolution of the ratio $M_{\rm PNS}/R_{\rm PNS}^2$ (the PNS surface gravity), where  $M_{\rm PNS}$ and $R_{\rm PNS}$ are the mass and the radius of the PNS.} Finally, using 2D CCSN waveform corresponding to different progenitor masses we estimate the performance of the algorithm for current and future generation of ground-based GW detectors.

This paper is organized as follows. Section II describes the details of the CCSN simulations used in the paper. The algorithm that we employ to extract the time evolution of the PNS surface gravity is discussed in Section III. Section IV shows the performance of our inference method with simulated GW detectors data and presents our main results. Our findings are summarized in section V. Appendix A discusses specific details related to the reconstruction of the g-mode.


