\section{Introduction}


\textbf{CCSN Simulation and GW emission}\\
The life of massive stars ($8 {\rm M}_\odot-100{\rm M}_\odot$) ends with the collapse of their iron core under their own gravity, leading to the formation of a neutron star or a black hole (BH), followed (typically but not necessarily in the BH case)  by the explosion of the star as a supernova. Core-collapse supernova (CCSN) explosions are one of the most important sources of gravitational-waves (GW) that have not yet been detected by current ground-based observatories. This is because even the most common type of CCSN, the neutrino-driven explosion supernova, have a rate of about three per century \cite{Gossan:2016} within our galaxy. The other main type of explosion, those produced by the magneto-rotational mechanism, can produce a more powerful signal and can be detected at distances up to $\sim 5$ Mpc \cite{Gossan:2016}. However, the rate of events of this kind is much lower than the one for the neutrino driven mechanism $\sim 10^{-4} \rm{yr}^{-1}$, which represents less than $1 \%$ of all CCSNe.
Despite all this, collapsing stars produces a complex GW signal which could provide significant clues about the physical processes that occur in the moments after the explosion. 

The computational modelling of the core-collapse phenomena is challenging due the large number of physical processes involved whose role is not completely understood. There are uncertainties in the stellar evolution of massive stars or in the nuclear and weak force interactions necessary for the equation of state (EoS) or the neutrino interactions.  These large number of physical ingredients in addition to the necessary accuracy of the modelling of complex multidimensional interactions requires large computational resources. One simulation of one single progenitor in 3D with accurate neutrino transport and realistic equation of state (EoS) can take several months of intense calculations on a scientific supercomputer facility.

\textbf{PNS oscillation modes}\\
\textbf{Asteroseismology and universal relationships}\\
\textbf{Introduce the topic of the paper: method to extract from the GW data the Mass and the radius of the PNS as function of time using the universal relations.}
\textbf{Previous work in identifying modes from spectrogram}\\
\textbf{Paper organization}\\

