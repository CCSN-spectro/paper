\section{Introduction}

The life of massive stars ($8 {\rm M}_\odot-100{\rm M}_\odot$) ends with the collapse of their iron core under their own gravity, leading the formation of a neutron star or a black hole (BH), followed (typically but not necessarily in the BH case)  by the explosion of the star as a supernova. Core-collapse supernova (CCSN) explosions are one most important sources of gravitational-waves (GW) that have not yet been detected by current ground-based observatories. This is because even the most common type of CCSN, the neutrino-driven explosion, have a rate about three per century \cite{Gossan:2016} within our galaxy. The other main type of explosion, those produced by the magneto-rotational mechanism, can produce a more powerful signal and can be detected at distances up to $\sim 5$ Mpc \cite{Gossan:2016}. However, the rate of events of this kind is much lower than the one for the neutrino driven mechanism $\sim 10^{-4} \rm{yr}^{-1}$, which represents less than $1 \%$ of all CCSNe.
Despite all this, collapsing stars produces a complex GW signal which could provide significant clues about the physical processes that occur in the moments after the explosion. 

In the past years an impressive progress has been made in the development of numerical codes, which allows to obtain more accurate CCSN simulations. The waveforms produced by the magneto-rotational mechanism in particular is well understood. The core-bounce signal can be directly related with the rotational properties of the core \cite{Dimmelmeier:2007, abdikamalov:14, Richers:2017}. However, the low rate of this kind of events and is low amplitude and high frequency of the bounce signal in the slow-rotation case will probably impede its detection.

In the case of most common neutrino-driven mechanism, the GW emission is manly produced during the hydrodynamical bounce and the unstable evolution of the fluid inside the region formed by the recently formed proto-neutron star (PNS) and the accretion shock. The dynamics excite the different modes of oscillation of the PNS \cite{kokkotas, Friedman:2013}. Unluckily, in this case is not posible to relate the GW emission with the properties (mass, rotation rate, metallicity or magnetic fields) of the progenitor stars. There are many reasons that explain this issue. The large number of physical processes involved whose role in the escenario is not completely understood. For instance, exist uncertainties in the stellar evolution of massive stars or in the nuclear and weak interactions necessary for the equation of state (EoS) or the neutrino interactions. Furthermore, the stochastic and chaotic nature of the instabilities is transferred to the GW emission, resulting in the same progenitor leading to a significantly different waveform.
These large number of physical ingredients in addition to the necessary accuracy of the modelling of complex multidimensional interactions requires large computational resources. One simulation of one single progenitor in 3D with accurate neutrino transport and realistic equation of state (EoS) can take several months of intense calculations on a scientific supercomputer facility, which complicates the systematic exploration of the progenitor parameters.

Common features in the GW signal, that have been interpreted as g modes of the PNS, have been reported in many works \cite{murphy:09, Cerda:2013, mueller:13gw, Yakunin:2015, Kuroda:2016, Andresen:2017}.  Typically, the frequencies associated with the modes rise monotonically with time during the contraction of the PNS. The characteristic frequencies of the modes associated to the PNS make them promising candidates for detection in ground-based interferometers.  


\textbf{Mode analysis and relations}\\

\textbf{This paper method description and previous works}\\
\textbf{Previous work in identifying modes from spectrogram}\\
\textbf{Paper organization}\\
