\section{Introduction}


\textbf{CCSN Simulation and GW emission}\\
The life of massive stars ($8 {\rm M}_\odot-100{\rm M}_\odot$) ends with the collapse of their iron core under their own gravity, leading the formation of a neutron star or a black hole (BH), followed (typically but not necessarily in the BH case)  by the explosion of the star as a supernova. Core-collapse supernova (CCSN) explosions are one most important sources of gravitational-waves (GW) that have not yet been detected by current ground-based observatories. This is because even the most common type of CCSN, the neutrino-driven explosion, have a rate about three per century \cite{Gossan:2016} within our galaxy. The other main type of explosion, those produced by the magneto-rotational mechanism, can produce a more powerful signal and can be detected at distances up to $\sim 5$ Mpc \cite{Gossan:2016}. However, the rate of events of this kind is much lower than the one for the neutrino driven mechanism $\sim 10^{-4} \rm{yr}^{-1}$, which represents less than $1 \%$ of all CCSNe.
Despite all this, collapsing stars produces a complex GW signal which could provide significant clues about the physical processes that occur in the moments after the explosion. 


\textbf{Mode analysis and relations}\\
\textbf{Martin s20 simulation and code description}\\
\textbf{This paper method description and previous works}\\
\textbf{Previous work in identifying modes from spectrogram}\\
\textbf{Paper organization}\\
