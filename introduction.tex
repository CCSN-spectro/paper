% !TEX root = ccsn.tex

\section{Introduction}


The life of massive stars \pcd{(those born with masses between $\sim 8$~M$_\odot$ and $\sim 120$~M$_\odot$)} ends with the collapse of \pcd{the} iron core under \pcd{its} own gravity, leading \pcd{to} the formation of a neutron star \pcd{(NS)} or a black hole (BH), \pcd{and} followed (typically but not necessarily in the BH case) by \pcd{a supernova} explosion. \pcd{Nearby} core-collapse supernova (CCSN) explosions \pcd{are expected to be sources of gravitational waves (GW) and are one of the main candidates for the next great discovery by current ground-based observatories. However, these are relative rare events. A neutrino-driven explosion \citep{Bethe:1990} is the most likely outcome in the case of slow rotating cores, which are present in the bulk of CCSN progenitors. This event could be detected with advanced GW detectors within $5$~kpc \citep{Gossan:2016,TargetedSNSearchO12}. Such a galactic event has a rate of about $2-3$ per century \citep{Adams:2013,Rozwadowska:2021}. For the case of fast rotating progenitor cores the result is likely a magneto-rotational explosion, with } a more powerful signal \pcd{that could} be detected \pcd{within $50$~kpc and for some extreme models 
up to $5-30$ Mpc \cite{Gossan:2016,TargetedSNSearchO12}. However, only about $1\%$ of the electromagnetically observed events show signatures of fast rotation (broad-lined type Ic SNe \citep{Li:2011b} or events associated to long GRBs 
\citep{Chapman:2007}), making this possibility a subdominant channel of detection with an event rate of $\sim 10^{-4} \rm{yr}^{-1}$. Therefore, we focus this work only in neutrino-driven CCSNe. 
Despite the low rates, CCSN are of great scientific interest because they produce} a complex GW signals which could provide significant clues about the physical processes that occur in the moments after the collapse. 

In the \pcd{last decade, a significant} progresses have been made in the development of numerical codes, \pcd{in particular in the treatment of multidimensioal effects \citep{BMueller:2020}.} \pcd{[PCD: I don't think this sentence is necessary since 
we have already discarded magneto-rotational events for this work. ]\sout{The waveforms produced by the magneto-rotational mechanism in particular is well understood. The core-bounce signal can be directly related with the rotational properties of the core} \cite{Dimmelmeier:2007, abdikamalov:14, Richers:2017}. \sout{However, the low rate of this kind of events and its expected low amplitude in the slow-rotation case, will probably impede its detection.}}

In the case of the neutrino-driven \pcd{explosions}, the GW emission is \pcd{primarly induced by instabilities developed at the newly formed proto-neutron star (PNS) and by the non-spherical accreting flow of hot matter over its surface.  This} dynamics excite the different modes of oscillation of the PNS\pcd{, which ultimately leads to the emission of GWs. [UNFINISHED! CONTINUE HERE]}Unluckily, in this case it is not posible to relate the GW emission with the properties (mass, rotation rate, metallicity or magnetic fields) of the progenitor stars.  A large number of physical processes are involved and their role is not completely understood. For instance,  uncertainties in the stellar evolution models of massive stars or in the nuclear and weak interactions necessary for the equation of state (EoS) of nuclear matter or the neutrino interactions. Furthermore, the stochastic and chaotic nature of the instabilities is transferred to the GW emission, resulting in the same progenitor leading significantly different waveforms.
The large number of physical ingredients in addition to the necessary accuracy of the modelling of complex multidimensional interactions requires large computational resources. One simulation of a single progenitor explosion  in 3D with accurate neutrino transport and realistic EoS can take several months of intense calculations on a scientific supercomputer facility. This complicates the systematic exploration of the progenitor parameters.

Common features in the GW signal, that have been interpreted as gravity modes (g-modes) oscillations of the PNS, have been reported in many articles \cite{murphy:09, Cerda:2013, mueller:13gw, Yakunin:2015, Kuroda:2016, Andresen:2017}. Typically, the frequencies associated with the modes rise monotonically with time during the contraction of the PNS. The characteristic frequencies of the modes associated to the PNS make them promising features for detection in ground-based interferometers. The presence of g-modes in hot PNS has been studied since the end of last century. The oscillation modes related with the surface of hot PNS was first considered by McDermott, van Horn \& Scholl \cite{McDermott:1983}. Additionally, the stratified structure of the PNS allows the presence of different types of g-modes related with the fluid core \cite{Reisenegger:1992}. Many posterior works used simplified neutron star models  assuming equilibrium configurations, to study the effect of rotation \cite{Ferrari:2004}, general relativity \cite{Passamonti:2005}, non-linearities \cite{Dimmelmeier:2006}, phase transition \cite{Kruger:2015} and realistic equation of state \cite{Camelio:2017}. Sotani \& Takiwaki \cite{Sotani:2016} studied the oscillation modes before the explosion using a simplified fits to numerical simulations.

\cite{kokkotas, Friedman:2013} In previous works \cite{Torres:2018, Torres:2019a}, we explore the eigenmode spectrum using results of CCSN numerical simulations and the theoretical model of the cavity form by the center of the PNS and the shock. We showed that the GW time-frequency distribution corresponds with the frequencies of oscillation of different families of p- and g-modes. These works reveal that is posible to perform CCSN asteroseismology and serves as a starting point to carry out inference of astrophysical parameters of PNSs. In this line of research, in \cite{Torres:2019b} we derived the relations between the different types of modes with some with the evolution of mass and radius of the PNS. These relations are universal in the sense that they not depend of the EOS, the mass of the progenitor or the code used to perform the simulation. 

In this paper, we present a method to extract from the GW data the mass and the radius of the PNS as function of time using the universal relations. We show how the algorithm is able to extract the time-frequency evolution of the main arc of GW emission, which corresponds to the $^2\rm{g}_2$ mode, according to the nomenclature used in \cite{Torres:2019b}. The universal relation for this mode is inverted to obtain the time evolution of the ratio $r=M_{\rm PNS}/R_{\rm PNS}^2$. Using 2D CCSN waveform corresponding to different progenitor masses we estimate teh performance of the algorithm for current and future generation of ground-based GW detectors.

This paper is organised as follows. Section II describes the details of the 2D CCSN used. Section III focuses on the algorithm that extracts the time evolution of a combination of the mass and radius of the PNS corresponding to a g-mode. Section IV shows the performance of the data analysis method for different GW detectors. Finally, we discuss the results in section V.


