\section{Introduction}


The life of massive stars ($8 {\rm M}_\odot-100{\rm M}_\odot$) ends with the collapse of their iron core under their own gravity, leading the formation of a neutron star or a black hole (BH), followed (typically but not necessarily in the BH case) by the explosion of the star as a supernova. Core-collapse supernova (CCSN) explosions are one of the expected sources of gravitational-waves (GW) that have not yet been detected by current ground-based observatories. This is because even the most common type of CCSN, the neutrino-driven explosion, have a rate about three per century \cite{Gossan:2016} within our galaxy. The other main type of explosion, the magneto-rotational mechanism, can produce a more powerful signal and can be detected at distances up to $\sim 5$ Mpc \cite{Gossan:2016} \textcolor{red} Add better reference. However, the rate of events of this kind is much lower than the one for the neutrino driven mechanism $\sim 10^{-4} \rm{yr}^{-1}$, which represents less than $1 \%$ of all CCSNe.
Despite all this, collapsing stars produces a complex GW signal which could provide significant clues about the physical processes that occur in the moments after the collapse. 

In the past years impressive progresses have been made in the development of numerical codes, which allow to obtain more accurate CCSN simulations. The waveforms produced by the magneto-rotational mechanism in particular is well understood. The core-bounce signal can be directly related with the rotational properties of the core \cite{Dimmelmeier:2007, abdikamalov:14, Richers:2017}. However, the low rate of this kind of events and its expected low amplitude in the slow-rotation case, will probably impede its detection.

In the case of the neutrino-driven explosion mechanism, the GW emission is mainly produced during the hydrodynamical bounce and the unstable evolution of the fluid inside the region formed by the recently formed proto-neutron star (PNS) and the accretion shock. The dynamics excite the different modes of oscillation of the PNS \cite{kokkotas, Friedman:2013}. Unluckily, in this case it is not posible to relate the GW emission with the properties (mass, rotation rate, metallicity or magnetic fields) of the progenitor stars.  A large number of physical processes are involved and their role is not completely understood. For instance,  uncertainties in the stellar evolution models of massive stars or in the nuclear and weak interactions necessary for the equation of state (EoS) of nuclear matter or the neutrino interactions. Furthermore, the stochastic and chaotic nature of the instabilities is transferred to the GW emission, resulting in the same progenitor leading significantly different waveforms.
The large number of physical ingredients in addition to the necessary accuracy of the modelling of complex multidimensional interactions requires large computational resources. One simulation of a single progenitor explosion  in 3D with accurate neutrino transport and realistic EoS can take several months of intense calculations on a scientific supercomputer facility. This complicates the systematic exploration of the progenitor parameters.

Common features in the GW signal, that have been interpreted as gravity modes (g-modes) oscillations of the PNS, have been reported in many articles \cite{murphy:09, Cerda:2013, mueller:13gw, Yakunin:2015, Kuroda:2016, Andresen:2017}. Typically, the frequencies associated with the modes rise monotonically with time during the contraction of the PNS. The characteristic frequencies of the modes associated to the PNS make them promising features for detection in ground-based interferometers. The presence of g-modes in hot PNS has been studied since the end of last century. The oscillation modes related with the surface of hot PNS was first considered by McDermott, van Horn \& Scholl \cite{McDermott:1983}. Additionally, the stratified structure of the PNS allows the presence of different types of g-modes related with the fluid core \cite{Reisenegger:1992}. Many posterior works used simplified neutron star models  assuming equilibrium configurations, to study the effect of rotation \cite{Ferrari:2004}, general relativity \cite{Passamonti:2005}, non-linearities \cite{Dimmelmeier:2006}, phase transition \cite{Kruger:2015} and realistic equation of state \cite{Camelio:2017}. Sotani \& Takiwaki \cite{Sotani:2016} studied the oscillation modes before the explosion using a simplified fits to numerical simulations.

In previous works \cite{Torres:2018, Torres:2019a}, we explore the eigenmode spectrum using results of CCSN numerical simulations and the theoretical model of the cavity form by the center of the PNS and the shock. We showed that the GW time-frequency distribution corresponds with the frequencies of oscillation of different families of p- and g-modes. These works reveal that is posible to perform CCSN asteroseismology and serves as a starting point to carry out inference of astrophysical parameters of PNSs. In this line of research, in \cite{Torres:2019b} we derived the relations between the different types of modes with some with the evolution of mass and radius of the PNS. These relations are universal in the sense that they not depend of the EOS, the mass of the progenitor or the code used to perform the simulation. 

In this paper, we present a method to extract from the GW data the mass and the radius of the PNS as function of time using the universal relations. We show how the algorithm is able to extract the time-frequency evolution of the main arc of GW emission, which corresponds to the $^2\rm{g}_2$ mode, according to the nomenclature used in \cite{Torres:2019b}. The universal relation for this mode is inverted to obtain the time evolution of the ratio $r=M_{\rm PNS}/R_{\rm PNS}^2$. Using 2D CCSN waveform corresponding to different progenitor masses we estimate teh performance of the algorithm for current and future generation of ground-based GW detectors.

This paper is organised as follows. Section II describes the details of the 2D CCSN used. Section III focuses on the algorithm that extracts the time evolution of a combination of the mass and radius of the PNS corresponding to a g-mode. Section IV shows the performance of the data analysis method for different GW detectors. Finally, we discuss the results in section V.


