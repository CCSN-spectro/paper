% !TEX root = ccsn.tex

\section{Introduction}


The life of massive stars {(those born with masses between $\sim 8$~M$_\odot$ and $\sim 120$~M$_\odot$)} ends with the collapse of {the} iron core under {its} own gravity, leading {to} the formation of a neutron star {(NS)} or a black hole (BH), {and} followed (typically but not necessarily in the BH case) by {a supernova} explosion. {Nearby} core-collapse supernova (CCSN) explosions {are expected to be sources of gravitational waves (GWs) and are one of the main candidates for the next great discovery by current ground-based observatories. However, these are relative rare events. A neutrino-driven explosion \citep{Bethe:1990} is the most likely outcome in the case of slow rotating cores, which are present in the bulk of CCSN progenitors.
\mab{The emitted GWs} could be detected with advanced ground-based GW detectors \mab{(
  Advanced LIGO\citep{TheLIGOScientific:2014jea},
  Advanced Virgo\citep{TheVirgo:2014hva} and
  KAGRA\citep{Aso:2013eba})} within $5$~kpc \citep{Gossan:2016,TargetedSNSearchO12}. Such a galactic event has a rate of about $2-3$ per century \citep{Adams:2013,Rozwadowska:2021}.
For the case of fast rotating progenitor cores the result is likely a magneto-rotational explosion, with } a more powerful \mab{GW} signal {that could} be detected {within $50$~kpc and for some extreme models 
up to $5-30$ Mpc \cite{Gossan:2016,TargetedSNSearchO12}. However, only about $1\%$ of the electromagnetically observed events show signatures of fast rotation (broad-lined type Ic SNe \citep{Li:2011b} or events associated to long GRBs 
\citep{Chapman:2007}), making this possibility a subdominant channel of detection with an event rate of $\sim 10^{-4} \rm{yr}^{-1}$\mab{add ref?}. Therefore, we focus this work only in neutrino-driven CCSNe. 
Despite the low rates, CCSN are of great scientific interest because they produce} a complex GW signals which could provide significant clues about the physical processes that occur in the moments after the collapse. 

In the last decade, significant progress has been made in the development of numerical codes, {in particular in the treatment of multidimensioal effects \citep{BMueller:2020}.}
% {[PCD: I don't think the next sentence is necessary since we have already discarded magneto-rotational events for this work.]
%\sout{The waveforms produced by the magneto-rotational mechanism in particular is well understood. The core-bounce signal can be directly related with the rotational properties of the core} \cite{Dimmelmeier:2007, abdikamalov:14, Richers:2017}. \sout{However, the low rate of this kind of events and its expected low amplitude in the slow-rotation case, will probably impede its detection.}}
In the case of  neutrino-driven explosions, the GW emission is {primarly induced by instabilities developed at the newly formed proto-neutron star (PNS) and by the non-spherical accreting flow of hot matter over its surface \citep{Kotake:2017}.  This} dynamics excite the different modes of oscillation of the PNS{, which ultimately leads to the emission of GWs. The frequency and time evolution of these modes carry information about the
properties of the GW emitter and could allow to perform PNS asteroseismology. }


%{[PCD: I would remove the discussion on the inference of the progenitor properties (commented now). ]}
%Unluckily, in this case it is not posible to relate the GW emission with the properties (mass, rotation rate, metallicity or magnetic fields) of the progenitor stars.  A large number of physical processes are involved and their role is not completely understood. For instance,  uncertainties in the stellar evolution models of massive stars or in the nuclear and weak interactions necessary for the equation of state (EoS) of nuclear matter or the neutrino interactions. Furthermore, the stochastic and chaotic nature of the instabilities is transferred to the GW emission, resulting in the same progenitor leading significantly different waveforms.
%The large number of physical ingredients in addition to the necessary accuracy of the modelling of complex multidimensional interactions requires large computational resources. One simulation of a single progenitor explosion  in 3D with accurate neutrino transport and realistic EoS can take several months of intense calculations on a scientific supercomputer facility. This complicates the systematic exploration of the progenitor parameters.


{The main feature appearing systematically in the GW spectrum of multidimensional numerical simulations is a strong and relatively narrow {\mab oscillation} in the post bounce evolution with raising frequency
from about $100$~Hz up to a few kHz (at most) and a typical duration of $0.5-1$~s. This feature has been interpreted as a continuously excited gravity mode (g-mode, see \citep{kokkotas,Friedman:2013} for a definition in this context) of the PNS \cite{murphy:09, mueller:13gw, Cerda:2013, Yakunin:2015, Kuroda:2016, Andresen:2017}. In these models the monotonic raise of the frequency of the mode is related to the contraction of the PNS.} The {typical} frequencies of {these} modes make them {a} promising {source} for ground-based interferometers. 
 
 
 The {properties of} g-modes in hot {PNSs} have been studied since the end of last century {by means of linear perturbation analysis of background PNS models}. The oscillation modes {associated to} the surface of hot PNSs was first considered by McDermott, van Horn \& Scholl \cite{McDermott:1983}. Additionally, the stratified structure of the PNS allows the presence of different types of g-modes related with the fluid core \cite{Reisenegger:1992}. Many posterior works used simplified neutron star models assuming an equilibrium configuration {as a background}, to study the effect of rotation \cite{Ferrari:2004}, general relativity \cite{Passamonti:2005}, non-linearities \cite{Dimmelmeier:2006}, phase transition \cite{Kruger:2015} and realistic equation of state \cite{Camelio:2017}. {Only recently, there has been an effort to incorporate realistic backgrounds based in numerical simulations in the computation of the mode structure and evolution \cite{Sotani:2016,Torres:2018, Morozova:2018, Torres:2019a,Torres:2019b,Sotani:2019,WS:2019,Sotani:2020a, Sotani:2020b}}.
 
 % {We base this work in the PNS mode analysis performed by \cite{Torres:2018, Torres:2019a}, which explored
\mab{The eigenmode spectrum of the region within the shock (including the PNS and the post-shock region) 
using results from 2D CCSN numerical simulations as a background studied in \cite{Torres:2018, Torres:2019a}
show a good match of the mode frequencies computed and the features observed in the GW spectrum of the same
simulation (specially when space-time perturbations are included \citep{Torres:2019a}).}
This result reveals that it is posible to perform CCSN asteroseismology {under realistic conditions} and serves as a starting point to carry out inference of astrophysical parameters of PNSs.  {\cite{Torres:2019b} went one step further showing that it was possible to derive simple relations} between the { instantaneous frequency of the g-mode and} the mass and radius of the PNS {at each time of the evolution}. These relations are universal in the sense that they do not depend {on the equation of state (EOS) used} or the mass of the progenitor, {and only weakly on} the numerical code used {(see discussion in section~\ref{sec:simulations})}. {Similar relations have been found by \cite{Sotani:2020a,Sotani:2020b}, which also found that the universal relations do not depend on the dimensionality (1D, 2D or 3D) of the numerical simulation used as a background.}
 
{In this work, we present a method to infer from the GW data alone, the time evolution of some properties or the PNS, namely a combination of its mass and radius. For this purpose we have developed an algorithm to  
extract the time-frequency evolution of the main feature in the spectrograms of the GW emission of 2D simulations of CCSN. This feature corresponds to the $^2\rm{g}_2$ mode, according to the nomenclature used in \cite{Torres:2019b} (different authors may have slightly different naming convention). Next, we use the universal relations obtained by \cite{Torres:2019b}{, based on a set of 1D simulations,} to infer the time evolution of the ratio $M_{\rm PNS}/R_{\rm PNS}^2$, being 
$M_{\rm PNS}$ and $R_{\rm PNS}$ the mass and radius of the PNS.} Using 2D CCSN waveform corresponding to different progenitor masses we estimate the performance of the algorithm for current and future generation of ground-based GW detectors.

This paper is organised as follows. Section II describes the details of the CCSN simulations used in the paper. Section III focuses on the algorithm that extracts the time evolution of a combination of the mass and radius of the PNS corresponding to a g-mode. Section IV shows the performance of the data analysis method with simulated GW detectors data. Finally, we discuss the results in section V.


